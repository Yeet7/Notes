\documentclass{report}

\input{preamble}
\input{macros}
\input{letterfonts}
\graphicspath{ {./images/PRECALC} }


\title{\Huge{Precalculus}}
\author{Me. I am Him.}
\date{11/28/2022}


\begin{document}

\maketitle
\newpage
\pdfbookmark[section]{\contentsname}{toc}
\tableofcontents
\pagebreak

\thispagestyle{empty}
\newpage
\listoffigures
\clearpage
\pagenumbering{arabic}
\pagebreak


\chapter{}

\chapter{}

\chapter{}
\section{Properties of functions and Complex Zeros}

\subsection{3.1 - Completing the square}
\dfn{Completing the square}{In form $ax^{2}+bx+c$, you half b, and then square.}

\nt{Forms:
	\begin{center}
		\item Standard: $ax^{2}+bx+c$ ; $a\neq0$
		\item Vertex: $a(x-h)^{2}+k$ ; $a\neq0$
	\end{center}}
\nt{Verticies:
	\begin{center}
		\item Standard: (-b/2a , f(-b/2a)) 
		\item Vertex: (h , k)
	\end{center}}
\nt{Axis of Symmetry:
	\begin{center}
		\item Standard: x=-b/2a 
		\item Vertex: x=h
	\end{center}}
\nt{y-Intercept
	\begin{center}
		\item Standard: (0 , c) 
		\item Vertex: \textit{Plug in 0 for x and solve} 
		\item Note;
		There is always one.
	\end{center}}
\nt{x-Intercept
	\begin{center}
		\item Standard: $x = \frac{-b\pm\sqrt{b^{2}-4ac}}{2a}$, otherwise known as the quadratic formula.
		\item If the inside of the quadratic is < 0, there are no x-intercepts.
		\item If the inside of the quadratic is = 0, there is exactly one x-intercept.
		\item If the inside of the quadratic is > 0, there are exactly two x-intercepts.
		\item Vertex: \textit{Plug in 0 for y and solve}
	\end{center}}




\chapter{4}
\section{4: Composite Functions}
\subsection{}
\nt{11/28/2022 - Didn't really do much. Just reviewed what $f \cdot g$ or f(g(x)) was.} 
\subsection{}

\subsection{Exponential Functions}
\arp{}{
	\begin{tabular}{rl}
		& $\bullet$ $4^{3} = 8$ \\
		& $\bullet$ $8^{\frac{2}{3}} = 4$ \\
		& $\bullet$ $3^{-2} = \frac{1}{9}$
	\end{tabular}}
\nt{In a$^{n}$, \textit{a} is known as the base whereas \textit{n} is known as the exponent, index, or power.}
\nt{Law of Exponents:
	\begin{enumerate}[label=(\arabic*)]
		\item $a^{m} \cdot a^{n} = a^{m+n}$			Example: 		$3^{2}\cdot 3^{5} = 3^{2+5} = 3^{7} = 2187$
		\item $(a^{m})^{n} = a^{mn}$		Example:		$(2^{3})^{2} = 2^{3\cdot 2} = 2^{6} = 64$
		\item $(ab)^{m} = a^{m}b^{m}$		Example:		$(5x)^3$
		\item $1^{n} = 1$		Example:			$1^{1001} = 1$
		\item $a^{-n} = \frac{1}{a^{n}}$		Example:		$5^{-2} = \frac{1}{5^{2}} = \frac{1}{25}$
		\item $a^{0} = 1$		Example:		$7^{0} = 1$
		\item $a^{\frac{m}{n}} = \sqrt[n]{a^{m}} = \left(\sqrt[n]{a}\right)^{m}$		Example:		$8^{\frac{2}{3}} = \sqrt[3]{8^{2}} = \left(\sqrt[3]{8}\right)^{2} = 2^{2} = 4$
	\end{enumerate}}
\dfn{Exponential Function}{A function of the form f(x) = a$^{x}$ where x is a positive real number (a>0) and a $\neq$ 1. The domain of $f$ is $\mathbb{R}$}
\ex{2: Graph the exponential function: $f(x) = 2^{x}$}{}
	\begin{figure}[ht]
		\centering
		\caption{$f(x) = 2^{x}$}
		\label{fig:101}
		\includegraphics[scale=0.25]{2^(x)}
	\end{figure}
\nt{Properties of the Exponential Function: $f(x) = a^{x}$, where $a>1$
	\begin{enumerate}
		\item The \textit{domain} is the set of all real numbers.
		The \textit{range} is the set of all positive real numbers.
		\item There are no \textit{x-intercepts}.
		The \textit{y-intercept} is 1.
		\item The \textit{x-axis} (y=0) is a horizontal asymptote as $x \rightarrow -\infty$
		\item The function is an increasing function and is one-to-one.
		\item The graph of f contains the points (0,1),(1,a), and (-1,1/a).
		\item The graph of f is smooth and continuous, with no corners or gaps.
	\end{enumerate}}
\ex{3: Graph the exponential function: $f(x)=\left(\frac{1}{2}\right)^{2}$}{}
\begin{figure}[ht]
	\centering
	\caption{$f(x)=\left(\frac{1}{2}\right)^{2}$}
	\label{fig:102}
	\includegraphics[scale=0.25]{0.5^(x)}
\end{figure}
\nt{Properties of Exponential Function: $f(x)=a^{x}$, where $0<x<1$.
\begin{enumerate}
	\item The domain is the set of all real numbers;
	the range is the set of positive real numbers.
	\item There are no x-intercepts;
	the y-intercept is 1.
	\item The x-axis (y = 0) is a horizontal asymptote as $x\rightarrow\infty$.
	\item The function is a decreasing function and is one-to-one.
	\item The graph of f contains the points (0, 1), (1, a), and (-1, 1/a).
	\item The graph of f is smooth and continuous, with no corners or gaps.
\end{enumerate}}
\ex{Graph $f(x) = 2^{-x}-3$ and determine the domain, range, and horizontal asymptote of $f$.}{
	\begin{enumerate}
		\item [-] Domain: ${x|x\in\mathbb{R}}$ or $[-\infty,\infty]$
		\item [-] Range: ${y|y>-3}$ or $[-3,\infty]$
		\item [-] Horizontal Asymptote: $y=-3$
\end{enumerate}
}
\begin{figure}[ht]
	\centering
	\caption{$f(x) = 2^{-x}-3$}
	\label{fig:103}
	\includegraphics[scale=0.25]{2^(-x)-3}
\end{figure}

\ex{Explain the transformation of g(x) from $f(x) = e^{x}$}{
	\begin{tabular}{rl} 
		& $\bullet$ $g(x) = -e^{x-3}$ \\
		& $\bullet$ $g(x) = 3e^{-x}-5$
	\end{tabular}}
\ex{6: Solve $3^{x+1}=81$}{
	\begin{tabular}{rl}
		& $\bullet$ $3^{x+1} = 3^{4}$ \\
		& $\bullet$ $x+1 = 4$ \\ 
		& $\bullet$ $x = 3$
	\end{tabular}}
\ex{7: Solve $e^{-x^2} = \left(e^{x^2}\cdot\frac{1}{e^{3}}\right)$}{
	\begin{tabular}{rl}
		& $\bullet$ $e^{-x^2} = e^{2x} \cdot e^{-3}$ \\
		& $\bullet$ $e^{-x^{2}} = e^{2x-3}$ \\
		& $\bullet$ $-x^{2} = 2x-3$ \\
		& $\bullet$ $x^{2}+2x-3$ \\
		& $\bullet$ $(x+3)(x-1) = 0$ \\
		& $\bullet$ $x=-3,1$
	\end{tabular}}
\ex{8: Between 9 AM and 10 PM cars arrive at burger king's drive-thru at the rate of 12 cars per hour (0.2 cars per minute). The following formula from statistics can be used to determine the probability that a car will arrive within t minutes of 9 PM}{
	$F(t)=1-e^{-2t}$
	\begin{enumerate}[label=(\alph*)]
		\item $63\%$
		\item $99.7\%$
		\item graph
		\item other thing
	\end{enumerate}}
\section{}
\section{}

\section{Logarithmic Functions}

\dfn{Logarithmic Function:}{The opposite to an exponential function. The logarithmic function to the base a, where $a>0$ and $a\neq0$, is denoted and defined by $y=\log_{x}x$ if and only if $x=a^{y}$}
\nt{You can remember the format by thinking log-base-answer-exponent.}

\ex{2: Change each exponential expression to an equivalent expression involving a logarithm.}{
	\begin{enumerate}
		\item $1.2^{3} \rightarrow $
	\end{enumerate}
}

\ex{3: Change each logarithmic expression to an equivalent expression involving an exponent.}{
	\begin{enumerate}
		\item $\log_{a}4=5 \rightarrow a^{5}=4$
		\item $\log_{b}e=-3 \rightarrow b^{-3}=e$
		\item $\log_{3}5=c \rightarrow 3^{c}=5$
	\end{enumerate}
}
\thm{}{
	Get that exponential theorem from slides}

\ex{4: Find he exact value of:}{
	\begin{enumerate}
		\item $\log_{2}16=x \rightarrow x=4$
		\item $\log_{3}\frac{1}{27}=x \rightarrow x=-3$ Convert to exponential then use the rules of exponents.
		\item $\log_{4}2=x \rightarrow x=\frac{1}{2}$
	\end{enumerate}
}

\thm{Determine the Domain of a logarithmic function:}{
	\begin{enumerate}
		\item [-] Domain of the logarithmic function = range of the exponential function = $(0,\infty)$
		\item [-] Range of the logarithmic function = domain of the exponential function = $(-\infty,\infty)$
	\end{enumerate}
}

\ex{5: Find the domain of each logarithmic function:}{
	\begin{enumerate}
		\item $f$($x$) = $\log_{2}(x+3) \rightarrow	x+3>0$
		\begin{itemize}
			\item $x>-3$ or $(-3,\infty)$
		\end{itemize}
		\item $g$($x$) = $\log_{b}\left(\frac{1+x}{1-x}\right) \rightarrow \frac{1+x}{1-x}>0$
		\begin{itemize}
			\item $x\neq 1,-1$.
			Now use a number line to find out where it applies.
			In this case it is $-1<x<1$ or $(-1,1)$ or ${x|x\neq 1,-1}$
		\end{itemize}
		\item $h$($x$) = $\log_{\frac{1}{2}}\abs{x} \rightarrow \abs{x}>0$
		\begin{itemize}
			\item $\textbf{Domain} = \mathbb{R}$ where $x\neq 0$, or All Real Numbers where $x\neq 0$, or ${x|x\neq 0}$
		\end{itemize}
	\end{enumerate}
}


\subsection{Natural Logarithm}
\thm{If the base of a logarithmic function is the number e, then we have the natural logarithm function. That is,}{
\begin{enumerate}
	\item y = ln x if and only if x = ey
	\item y = ln x and y = ex are inverse functions	
\end{enumerate}}
\thm{Common Logarithm Function}{If the base of a logarithmic function is the number 10, then we have the common logarithm function. If the base a of the logarithmic function is not indicated, it is understood to be 10. That is,
\begin{itemize}
	\item $y=log_{x}$ if and only if $x=10^y$
\end{itemize}}

\ex{6 \& 7: Determine the domain, range, and vertical asymptote of each logarithmic function. List any transformations.}{
	\begin{enumerate}
		\item [a.] $f(x) = \ln(x) \rightarrow g(x) = -\ln(x+2)$
		\begin{itemize}
			\item Domain: $x>-2$
			\item Range: $(-\infty,\infty)$
			\item Vertical Asymptote: $x\neq-2$
		\end{itemize} 
		\nt{The negtive applied to the natural log, seen in the equation $-\ln(x+2)$, is causing it to reflect over the x-axis.}
		\item [b.] $f(x) = \log(x) \rightarrow g(x) = 3\log(-x)-1$
	\end{enumerate}
}
\thm{Equations that contain logarithms are called logarithmic equations. Be sure to check each solution in the original equation and discard any extraneous solutions. Remember in logaM, a and M are positive and $a\neq 1$.
\begin{enumerate}
	\item Change the logarithmic equation to an exponential equation and solve for x
	\item If the exponential equation has base e, change it to the natural logarithm function
	\item If the exponential equation has base 10, change it to the common logarithm function
\end{enumerate}}

\ex{8: Solve for x}{
	\begin{enumerate}
		\item $\log_{3}(4x-7)=2$
		\begin{itemize}
			\item $3^{2}=4x-7$
			\item $9=4x-7$
			\item $x=4$
		\end{itemize}
		\item $\log_{x}(64)=2$
		\begin{itemize}
			\item $x^{2}=64$
			\item $x=\sqrt[2]{64}$
			\item $x=8$ Note: -8 does not work as a solution as base values for a logarithm must be greater than 1.
		\end{itemize}
	\end{enumerate}
}

\ex{8.5: Solve for x. Give the exact solution then use your calculator to give the approximate solution.}{
	\begin{enumerate}
		\item $e^{2x}=5$
		\begin{itemize}
			\item $\log_{e}5=2x$
			\item $\ln5=2x$
			\item $\frac{\ln5}{2} = x$
		\end{itemize}
	\end{enumerate}
}

\ex{Additional Example:}{
	\begin{enumerate}
		\item $10^{x^{2}+2x+1}=50$
		\begin{itemize}
			\item $\log(50)=x^{2}+2x+1$
			\item $\pm\sqrt[]{\log(50)} = \sqrt[]{(x+1)^{2}}$
			\item $\pm\sqrt[]{\log(50)}=x+1$
			\item $x=\pm\sqrt[]{\log(50)}+1$
		\end{itemize}
	\end{enumerate}
}

\begin{figure}
	\centering
	\caption{$\log_{10}x$}
	\label{fig:log_10(x)}
	\includegraphics[scale=0.25]{log_(10)x}
\end{figure}

\ex{10: The concentration of alcohol in a person's blood is measurable. 
Recent medical research suggests that the risk $R$ (given as a percent) of having an accident while driving a car can be modeled by the equation $6e^{kx}$ where x is the variable concentration of alcohol n the blood and k is a constant.
\begin{enumerate}
	\item Suppose that a concentration in the blood of 0.04 results in a 10\% risk (R=10) of an accident.
	Find the constant k in the equation.
	Graph $R=6e^{kx}$ using the k value.
	\begin{itemize}
		\item do stuff so that k=20.62.
		She literally used her calc
	\end{itemize}
	\item Using the value of k, what is the risk if the concentration is 0.17?
	\begin{itemize}
		\item uhhhh she didn't do this.
	\end{itemize}
	\item Using the same value of k, what concentration of alcohol corresponds to a risk of 100\%?
	\begin{itemize}
		\item didn't do this one either.
		apparently D is the most important.
	\end{itemize}
	\item If the law asserts that anyone with a risk of having an accident of 20\% or more should not have driving privileges, at what concentration of alcohol in the blood should a driver be arrested and charged with a DUI?
	\begin{itemize}
		\item $20=6e^{12.77x}$
		\item $\frac{10}{3}=e^{12.77x}$
		\item $\ln\left(\frac{10}{3}\right)=12.77x$
		\item $x=0.94$
	\end{itemize}
\end{enumerate}}

\section{4.5: Properties of Logarithms}

\nt{Properties of logarithms:
\begin{enumerate}
	\item Identity $\rightarrow \log_{a}1=0$ or $\log_{a}a=1$
	\item Inverse $\rightarrow \log_{b}b^{x}=x$ or $b^{\log_{b}(x)}x$
	\item Product $\rightarrow \log_{a}xy=\log = \log_{a}x+\log_{a}y$
	\item Quotient $\rightarrow \log_{a}\frac{x}{y}=\log_{a}-\log_{a}y$
	\item Equality $\rightarrow \log_{b}a=\log_{b}c \Rrightarrow a=c$
	\item Change of Base Formula $\rightarrow \log_{a}b=\frac{\log_{c}b}{\log_{c}b}$
\end{enumerate}}

\ex{1: Use properties of logarithms to find the exact value of each expression. Do not use a calculator.}{
	\begin{enumerate}
		\item $\ln e^{\sqrt{2}}$
		\begin{itemize}
			\item $\log_{e}e^{\sqrt{2}}$
			\item $\sqrt{2}\times\log_{e}e$
			\item $\sqrt{2}\times\ln{e}$
		\end{itemize}
		\item $\log_{8}16-\log_{8}2$
		\begin{itemize}
			\item $\log_{8}\frac{16}{2}$
			\item $\log_{8}8$
			\item $1$
		\end{itemize}
	\end{enumerate}
}
\ex{3:Write the expression as a sum of logarithms. Express all powers as factors.}{
	\begin{enumerate}
		\item $\log_{a}\left(x\sqrt{x^{2}+1}\right)$
		\begin{itemize}
			\item $\log_{a}x + \log_{a}\sqrt{x^{2}+1}$
			\item $\log_{a}x + \log_{a}(x^{2}+1)^{\frac{1}{2}}$
			\item $\log_{a}x + \frac{1}{2}\log_{a}(x^{2}+1)$
		\end{itemize}
	\end{enumerate}
}
\ex{4: Write the expression as a difference in logarithms. Express all powrs as factors.}{
	\begin{enumerate}
		\item $\ln\left(\frac{x^{2}}{(x-1)^{3}}\right)$
		\begin{itemize}
			\item $\ln(x^{2})-\ln(x-1)^{3}$
			\item $2\ln(x)-3\ln(x-1)$
		\end{itemize}
	\end{enumerate}
}
\ex{6: Write each of the following as a single logarithm.}{
	\begin{enumerate}
		\item $\log_{a}7+4\log_{a}3$
		\begin{itemize}
			\item $\log_{a}7+\log_{a}3^{4}$
			\item $\log_{a}(7\times 3^{4})$
			\item $\log_{a}567$
		\end{itemize}
		\item $\frac{2}{3}\ln8-\ln(3^{4}-8)$
		\begin{itemize}
			\item $\ln8^{\frac{2}{3}}-\ln(3^{4}-8)$
			\item $\ln\left(\frac{8^{\frac{2}{3}}}{3^{4}-8}\right)$
			\item $\ln\left(\frac{4}{7^{3}}\right)$
		\end{itemize}
		\item $\log _a x+\log _a 9+\log _a\left(x^2+1\right)-\log _a 5$
		\begin{itemize}
			\item blah
		\end{itemize}
	\end{enumerate}
}
\ex{7: Approximate the following. Round answers to four decimal places.}{
	\begin{enumerate}
		\item $\log_{2}27$
		\begin{itemize}
			\item $\frac{log_{10}27}{\log_{10}2}$ Note: 10 is the common base, thus it can be omitted.
			\item Another answer could be, $\frac{\ln27}{\ln2}$
		\end{itemize}
	\end{enumerate}
}
\ex{9: Use a graphing utility to graph the following}{
	\begin{enumerate}
		\item $y=\log_{2}x$
	\end{enumerate}
}
\begin{figure}
	\caption{$\log_{2}x$}
	\label{fig:log_(2)x}
	\includegraphics[scale=0.25]{log_(2)x}
\end{figure}















\newpage
\begin{figure}
	\centering
	\Huge{Thanks for reading}
	\includegraphics[scale=0.2]{dabloon}
	\label{fig:dabloonia}
\end{figure}




\end{document}
