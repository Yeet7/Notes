\documentclass{report}

\usepackage{graphicx}
	\graphicspath{ {./images/Precalc} }
\usepackage[T1]{fontenc}
\usepackage{enumitem}
\usepackage{pgfplots}
	\pgfplotsset{compat=1.18}
\usepackage{bold-extra}
\usepackage{lmodern}
\usepackage{hyperref}
\usepackage{array}
\input{preamble}
\input{macros}
\input{letterfonts}


\title{\Huge{Precalculus}}
\author{Me. I am Him.}
\date{11/28/2022}

\begin{document}

\maketitle
\newpage
\pdfbookmark[section]{\contentsname}{toc}
\tableofcontents
\pagebreak

\chapter{}
\chapter{}

\chapter{}
\section{Properties of functions and Complex Zeros}

\subsection{3.1 - Completing the square}
\dfn{Completing the square}{In form $ax^{2}+bx+c$, you half b, and then square.}

\nt{Forms:
	\begin{center}
		\item Standard: $ax^{2}+bx+c$ ; $a\neq0$
		\item Vertex: $a(x-h)^{2}+k$ ; $a\neq0$
	\end{center}}
\nt{Verticies:
	\begin{center}
		\item Standard: (-b/2a , f(-b/2a)) 
		\item Vertex: (h , k)
	\end{center}}
\nt{Axis of Symmetry:
	\begin{center}
		\item Standard: x=-b/2a 
		\item Vertex: x=h
	\end{center}}
\nt{y-Intercept
	\begin{center}
		\item Standard: (0 , c) 
		\item Vertex: \textit{Plug in 0 for x and solve} 
		\item Note; There is always one.
	\end{center}}
\nt{x-Intercept
	\begin{center}
		\item Standard: $x = \frac{-b\pm\sqrt{b^{2}-4ac}}{2a}$, otherwise known as the quadratic formula.
		\item If the inside of the quadratic is < 0, there are no x-intercepts.
		\item If the inside of the quadratic is = 0, there is exactly one x-intercept.
		\item If the inside of the quadratic is > 0, there are exactly two x-intercepts.
		\item Vertex: \textit{Plug in 0 for y and solve}
	\end{center}}




\chapter{}
\section{4: Composite Functions}
\subsection{}
\nt{11/28/2022 - Didn't really do much. Just reviewed what $f \cdot g$ or f(g(x)) was.} 
\subsection{}

\subsection{Exponential Functions}
\dfn{Are you prepared? MAKE NEW COMMAND}{
	\begin{tabular}{rl}
		& $\bullet$ $4^{3} = 8$ \\
		& $\bullet$ $8^{\frac{2}{3}} = 4$ \\
		& $\bullet$ $3^{-2} = \frac{1}{9}$
	\end{tabular}}
\nt{In a$^{n}$, \textit{a} is known as the base whereas \textit{n} is known as the exponent, index, or power.}
\nt{Law of Exponents:
	\begin{enumerate}[label=(\arabic*)]
		\item $a^{m} \cdot a^{n} = a^{m+n}$			Example: 		$3^{2}\cdot 3^{5} = 3^{2+5} = 3^{7} = 2187$
		\item $(a^{m})^{n} = a^{mn}$		Example:		$(2^{3})^{2} = 2^{3\cdot 2} = 2^{6} = 64$
		\item $(ab)^{m} = a^{m}b^{m}$		Example:		$(5x)^3$
		\item $1^{n} = 1$		Exaple:			$1{1001} = 1$
		\item $a^{-n} = \frac{1}{a^{n}}$		Example:		$5^{-2} = \frac{1}{5^{2}} = \frac{1}{25}$
		\item $a^{0} = 1$		Example:		$7^{0} = 1$
		\item $a^{\frac{m}{n}} = \sqrt[n]{a^{m}} = (\sqrt[n]{a})^{m}$		Example:		$8^{\frac{2}{3}} = \sqrt[3]{8^{2}} = (\sqrt[3]{8})^{2} = 2^{2} = 4$
	\end{enumerate}}
\dfn{Exponential Function}{A function of the form f(x) = a$^{x}$ where x is a positive real number (a>0) and a $\neq$ 1. The domain of $f$ is $\mathbb{R}$}
\ex{2: Graph the exponential function: $f(x) = 2^{x}$}{
	\begin{tikzpicture}
		\begin{axis}[
			xlabel=$x$,
			ylabel=$f(x)$,
			xtick={-3,-2,-1,0,1,2,3,4,5},
			ytick={0,1,2,4,8,16,32},
			ymin=0,
			ymax=32,
			xmin=-3,
			xmax=5,
			yticklabels={0,1,2,4,8,16,32},
			xticklabels={-3,-2,-1,0,1,2,3,4,5},
		]
		\addplot[samples=200, smooth, domain=-3:5, red] {2^x};
		\end{axis}
		\end{tikzpicture}}
\nt{Properties of the Exponential Function: $f(x) = a^{x}$, where $a>1$
	\begin{enumerate}
		\item The \textit{domain} is the set of all real numbers. The \textit{range} is the set of all positive real numbers.
		\item There are no \textit{x-intercepts}. The \textit{y-intercept} is 1.
		\item The \textit{x-axis} (y=0) is a horizontal asymptote as $x \rightarrow -\infty$
		\item The function is an increasing function and is one-to-one.
		\item The graph of f contains the points (0,1),(1,a), and (-1,1/a).
		\item The graph of f is smooth and continuous, with no corners or gaps.
	\end{enumerate}}
\ex{3: Graph the exponential function: $f(x)=(\frac{1}{2})^{2}$}{
	\begin{tikzpicture}
		\begin{axis}[
			xlabel=$x$,
			ylabel=$f(x)$,
			xtick={-5,-4,-3,-2,-1,0,1,2,3,4,5},
			ytick={0,1,2,4,8,16,32},
			ymin=0,
			ymax=32,
			xmin=-5,
			xmax=5,
			yticklabels={0,1,2,4,8,16,32},
			xticklabels={-5,-4,-3,-2,-1,0,1,2,3,4,5},
		]
		\addplot[samples=200, smooth, domain=-5:5, red] {(1/2)^x};
		\end{axis}
	\end{tikzpicture}}
\nt{Properties of Exponential Function: $f(x)=a^{x}$, where $0<x<1$.
\begin{enumerate}
	\item The domain is the set of all real numbers; the range is the set of positive real numbers.
	\item There are no x-intercepts; the y-intercept is 1.
	\item The x-axis (y = 0) is a horizontal asymptote as $x\rightarrow\infty$.
	\item The function is an decreasing function and is one-to-one.
	\item The graph of f contains the points (0, 1), (1, a), and (-1, 1/a).
	\item The graph of f is smooth and continuous, with no corners or gaps.
\end{enumerate}}
\ex{Graph $f(x) = 2^{-x}-3$ and determine the domain, range, and horizontal asymptote of $f$.}{
	\begin{tikzpicture}
		\begin{axis}[
			xlabel=$x$,
			ylabel=$f(x)$,
			xtick={-5,-4,-3,-2,-1,0,1,2,3,4,5},
			ytick={-3,0,3,6,9,12,14,18,24,32},
			ymin=-3,
			ymax=32,
			xmin=-5,
			xmax=5,
			yticklabels={-3,0,3,6,9,12,14,18,24,32},
			xticklabels={-5,-4,-3,-2,-1,0,1,2,3,4,5},]
		\addplot[samples=200, smooth, domain=-5:5, red] {2^(-x)-3};
		\end{axis}
	\end{tikzpicture}
\begin{enumerate}
		\item [-] Domain: ${x|x\in\mathbb{R}}$ or $[-\infty,\infty]$
		\item [-] Range: ${y|y>-3}$ or $[-3,\infty]$
		\item [-] Horizontal Asymptote: $y=-3$
\end{enumerate}
}
\ex{Explain the transformation of g(x) from $f(x) = e^{x}$}{
	\begin{tabular}{rl} 
		& $\bullet$ $g(x) = -e^{x-3}$ \\
		& $\bullet$ $g(x) = 3e^{-x}-5$
	\end{tabular}}
\ex{6: Solve $3^{x+1}=81$}{
	\begin{tabular}{rl}
		& $\bullet$ $3^{x+1} = 3^{4}$ \\
		& $\bullet$ $x+1 = 4$ \\ 
		& $\bullet$ $x = 3$
	\end{tabular}}
\ex{7: Solve $e^{-x^2} = (e^{x^2}\cdot\frac{1}{e^{3}})$}{
	\begin{tabular}{rl}
		& $\bullet$ $e^{-x^2} = e^{2x} \cdot e^{-3}$ \\
		& $\bullet$ $e^{-x^{2}} = e^{2x-3}$ \\
		& $\bullet$ $-x^{2} = 2x-3$ \\
		& $\bullet$ $x^{2}+2x-3$ \\
		& $\bullet$ $(x+3)(x-1) = 0$ \\
		& $\bullet$ $x=-3,1$
	\end{tabular}}
\ex{8: Between 9 AM and 10 PM cars arrive at burger king's drive-thru at the rate of 12 cars per hour (0.2 cars per minute). The following formula from statistics can be used to determine the probability that a car will arrive within t minutes of 9 PM}{
	$F(t)=1-e^{-2t}$
	\begin{enumerate}[label=(\alph*)]
		\item $63\%$
		\item $99.7\%$
		\item graph
		\item other thing
	\end{enumerate}}

\section{Logarithmic Functions}

\dfn{Logarithmic Function:}{The opposite to an exponential function. The logarithmic function to the base a, where $a>0$ and $a\neq0$, is denoted and defined by $y=\log_{x}x$ if and only if $x=a^{y}$}
\nt{You can remember the format by thinking log-base-answer-exponent.}

\ex{2: Change each exponential expression to an equivalent expression involving a logarithm.}{
	\begin{enumerate}
		\item $1.2^{3} \rightarrow $
	\end{enumerate}
}

\ex{3: Change each logarithmic expression to an equivalent expression involving an exponent.}{
	\begin{enumerate}
		\item $\log_{a}4=5 \rightarrow a^{5}=4$
		\item $\log_{b}e=-3 \rightarrow b^{-3}=e$
		\item $\log_{3}5=c \rightarrow 3^{c}=5$
	\end{enumerate}
}
\begin{theorem}
	Get that exponential theorem from slides
\end{theorem}

\ex{4: Find he exact value of:}{
	\begin{enumerate}
		\item $\log_{2}16=x \rightarrow x=4$
		\item $\log_{3}\frac{1}{27}=x \rightarrow x=-3$ Convert to exponential then use the rules of exponents.
		\item $\log_{4}2=x \rightarrow x=\frac{1}{2}$
	\end{enumerate}
}

\thm{Determine the Domain of a logarithmic function:}{
	\begin{enumerate}
		\item [-] Domain of the logarithmic function = range of the exponential function = $(0,\infty)$
		\item [-] Range of the logarithmic function = domain of the exponential function = $(-\infty,\infty)$
	\end{enumerate}
}

\ex{5: Find the domain of each logarithmic function:}{
	\begin{enumerate}
		\item $f(x) = \log_{2}(x+3) \rightarrow	x+3>0 \rightarrow x>-3 \rightarrow (-3,\infty)$
		\item $g(x) = \log_{b}(\frac{1+x}{1-x}) \rightarrow \frac{1+x}{1-x}>0 \rightarrow x\neq 1,-1$. Now use a number line to find out where it applies. In this case it is $-1<x<1$ or $(-1,1)$ or ${x|x\neq 1,-1}$
		\item $h(x) = \log_{\frac{1}{2}}\abs{x} \rightarrow \abs{x}>0 \rightarrow \textbf{Domain} = \mathbb{R}$ where $x\neq 0$, or All Real Numbers where $x\neq 0$, or ${x|x\neq 0}$
	\end{enumerate}
}



\newpage
\huge{Graphs}
\begin{tikzpicture}
	\begin{axis}[
		xlabel=$x$,
		ylabel=$y$,
		xtick={-5,-4,-3,-2,-1,0,1,2,3,4,5},
		ytick={-5,-4,-3,-2,-1,0,1,2,3,4,5},
		ymin=-5,
		ymax=3,
		xmin=-3,
		xmax=5,
		yticklabels={5,-4,-3,-2,-1,0,1,2,3,4,5},
		xticklabels={-5,-4,-3,-2,-1,0,1,2,3,4,5},]
	\addplot[samples=5000, smooth, domain=-5:5, red] {log10(x)};
	\end{axis}
\end{tikzpicture}







\newpage
\begin{figure}
	\centering
	\Huge{Thanks for reading}
	\includegraphics[scale=0.2]{dabloon.jpg}
	\label{dabloonia}
\end{figure}




\end{document}