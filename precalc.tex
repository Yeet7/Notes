\documentclass{report}
\usepackage{enumerate}

\input{preamble}
\input{macros}
\input{letterfonts}
\graphicspath{{./images/PRECALC}}


\title{\Huge{Precalculus}}
\author{Me. I am Him.}
\date{11/28/2022}


\begin{document}

\maketitle
\newpage
\pdfbookmark[section]{\contentsname}{toc}
\tableofcontents
\pagebreak

\thispagestyle{empty}
\newpage
\listoffigures
\clearpage
\pagenumbering{arabic}
\pagebreak


\chapter{Placeholder1}

\chapter{Placeholder2}

\chapter{Placeholder3}
\section{Properties of functions and Complex Zeros}

\subsection{3.1 - Completing the square}
\dfn{Completing the square}{In form $ax^{2}+bx+c$, you half b, and then square.}

\nt{Forms:
	\begin{center}
		\item Standard: $ax^{2}+bx+c$ ; $a \neq 0$
		\item Vertex: $a(x-h)^{2}+k$ ; $a \neq 0$
	\end{center}}
\nt{Verticies:
	\begin{center}
		\item Standard: (-b/2a , f(-b/2a)) 
		\item Vertex: (h , k)
	\end{center}}
\nt{Axis of Symmetry:
	\begin{center}
		\item Standard: x=-b/2a 
		\item Vertex: x=h
	\end{center}}
\nt{y-Intercept
	\begin{center}
		\item Standard: (0 , c) 
		\item Vertex: \textit{Plug in 0 for x and solve} 
		\item Note;
		There is always one.
	\end{center}}
\nt{x-Intercept
	\begin{center}
		\item Standard: $x = \frac{-b\pm\sqrt{b^{2}-4ac}}{2a}$, otherwise known as the quadratic formula.
		\item If the inside of the quadratic is < 0, there are no x-intercepts.
		\item If the inside of the quadratic is = 0, there is exactly one x-intercept.
		\item If the inside of the quadratic is > 0, there are exactly two x-intercepts.
		\item Vertex: \textit{Plug in 0 for y and solve}
	\end{center}}




\chapter{Exponential and Logarithmic Functions}
\section{4: Composite Functions}
\subsection{}
\nt{11/28/2022 - Didn't really do much. Just reviewed what $f \cdot g$ or f(g(x)) was.} 
\subsection{}

\subsection{Exponential Functions}
\arp{}{
	\begin{tabular}{rl}
		& $\bullet$ $4^{3} = 8$ \\
		& $\bullet$ $8^{\frac{2}{3}} = 4$ \\
		& $\bullet$ $3^{-2} = \frac{1}{9}$
	\end{tabular}}
\nt{In a$^{n}$, \textit{a} is known as the base whereas \textit{n} is known as the exponent, index, or power.}
\nt{Law of Exponents:
	\begin{enumerate}[label=(\arabic*)]
		\item $a^{m} \cdot a^{n} = a^{m+n}$			Example: 		$3^{2}\cdot 3^{5} = 3^{2+5} = 3^{7} = 2187$
		\item $(a^{m})^{n} = a^{mn}$		Example:		$(2^{3})^{2} = 2^{3\cdot 2} = 2^{6} = 64$
		\item $(ab)^{m} = a^{m}b^{m}$		Example:		$(5x)^3$
		\item $1^{n} = 1$		Example:			$1^{1001} = 1$
		\item $a^{-n} = \frac{1}{a^{n}}$		Example:		$5^{-2} = \frac{1}{5^{2}} = \frac{1}{25}$
		\item $a^{0} = 1$		Example:		$7^{0} = 1$
		\item $a^{\frac{m}{n}} = \sqrt[n]{a^{m}} = \left(\sqrt[n]{a}\right)^{m}$		Example:		$8^{\frac{2}{3}} = \sqrt[3]{8^{2}} = \left(\sqrt[3]{8}\right)^{2} = 2^{2} = 4$
	\end{enumerate}}
\dfn{Exponential Function}{A function of the form f(x) = a$^{x}$ where x is a positive real number (a>0) and a $\neq$ 1. The domain of $f$ is $\mathbb{R}$}
\ex{2: Graph the exponential function: $f(x) = 2^{x}$}{}
	\begin{figure}[ht]
		\centering
		\caption{$f(x) = 2^{x}$}
		\label{fig:101}
		\includegraphics[scale=0.25]{2^(x)}
	\end{figure}
\nt{Properties of the Exponential Function: $f(x) = a^{x}$, where $a>1$
	\begin{enumerate}
		\item The \textit{domain} is the set of all real numbers.
		The \textit{range} is the set of all positive real numbers.
		\item There are no \textit{x-intercepts}.
		The \textit{y-intercept} is 1.
		\item The \textit{x-axis} (y=0) is a horizontal asymptote as $x \rightarrow -\infty$
		\item The function is an increasing function and is one-to-one.
		\item The graph of f contains the points (0,1),(1,a), and (-1,1/a).
		\item The graph of f is smooth and continuous, with no corners or gaps.
	\end{enumerate}}
\ex{3: Graph the exponential function: $f(x)=\left(\frac{1}{2}\right)^{2}$}{}
\begin{figure}[ht]
	\centering
	\caption{$f(x)=\left(\frac{1}{2}\right)^{2}$}
	\label{fig:102}
	\includegraphics[scale=0.25]{0.5^(x)}
\end{figure}
\nt{Properties of Exponential Function: $f(x)=a^{x}$, where $0<x<1$.
\begin{enumerate}
	\item The domain is the set of all real numbers;
	the range is the set of positive real numbers.
	\item There are no x-intercepts;
	the y-intercept is 1.
	\item The x-axis (y = 0) is a horizontal asymptote as $x\rightarrow\infty$.
	\item The function is a decreasing function and is one-to-one.
	\item The graph of f contains the points (0, 1), (1, a), and (-1, 1/a).
	\item The graph of f is smooth and continuous, with no corners or gaps.
\end{enumerate}}
\ex{Graph $f(x) = 2^{-x}-3$ and determine the domain, range, and horizontal asymptote of $f$.}{
	\begin{enumerate}
		\item [-] Domain: ${x|x\in\mathbb{R}}$ or $[-\infty,\infty]$
		\item [-] Range: ${y|y>-3}$ or $[-3,\infty]$
		\item [-] Horizontal Asymptote: $y=-3$
\end{enumerate}
}
\begin{figure}[ht]
	\centering
	\caption{$f(x) = 2^{-x}-3$}
	\label{fig:103}
	\includegraphics[scale=0.25]{2^(-x)-3}
\end{figure}

\ex{Explain the transformation of g(x) from $f(x) = e^{x}$}{
	\begin{tabular}{rl} 
		& $\bullet$ $g(x) = -e^{x-3}$ \\
		& $\bullet$ $g(x) = 3e^{-x}-5$
	\end{tabular}}
\ex{6: Solve $3^{x+1}=81$}{
	\begin{tabular}{rl}
		& $\bullet$ $3^{x+1} = 3^{4}$ \\
		& $\bullet$ $x+1 = 4$ \\ 
		& $\bullet$ $x = 3$
	\end{tabular}}
\ex{7: Solve $e^{-x^2} = \left(e^{x^2}\cdot\frac{1}{e^{3}}\right)$}{
	\begin{tabular}{rl}
		& $\bullet$ $e^{-x^2} = e^{2x} \cdot e^{-3}$ \\
		& $\bullet$ $e^{-x^{2}} = e^{2x-3}$ \\
		& $\bullet$ $-x^{2} = 2x-3$ \\
		& $\bullet$ $x^{2}+2x-3$ \\
		& $\bullet$ $(x+3)(x-1) = 0$ \\
		& $\bullet$ $x=-3,1$
	\end{tabular}}
\ex{8: Between 9 AM and 10 PM cars arrive at burger king's drive-thru at the rate of 12 cars per hour (0.2 cars per minute). The following formula from statistics can be used to determine the probability that a car will arrive within t minutes of 9 PM}{
	$F(t)=1-e^{-2t}$
	\begin{enumerate}[label=(\alph*)]
		\item $63\%$
		\item $99.7\%$
		\item graph
		\item other thing
	\end{enumerate}}
\section{}
\section{}

\section{Logarithmic Functions}

\dfn{Logarithmic Function:}{The opposite to an exponential function. The logarithmic function to the base a, where $a>0$ and $ a \neq 0$, is denoted and defined by $y= \log_{x}x$ if and only if $x=a^{y}$}
\nt{You can remember the format by thinking log-base-answer-exponent.}

\ex{2: Change each exponential expression to an equivalent expression involving a logarithm.}{
	\begin{enumerate}
        \item $1.2^{3} \rightarrow $
	\end{enumerate}
}

\ex{3: Change each logarithmic expression to an equivalent expression involving an exponent.}{
	\begin{enumerate}
		\item $\log_{a}4=5 \rightarrow a^{5}=4$
		\item $\log_{b}e=-3 \rightarrow b^{-3}=e$
		\item $\log_{3}5=c \rightarrow 3^{c}=5$
	\end{enumerate}
}
\thm{}{
	Get that exponential theorem from slides}

\ex{4: Find he exact value of:}{
	\begin{enumerate}
		\item $\log_{2}16=x \rightarrow x=4$
		\item $\log_{3}\frac{1}{27}=x \rightarrow x=-3$ Convert to exponential then use the rules of exponents.
		\item $\log_{4}2=x \rightarrow x=\frac{1}{2}$
	\end{enumerate}
}

\thm{Determine the Domain of a logarithmic function:}{
	\begin{enumerate}
		\item [-] Domain of the logarithmic function = range of the exponential function = $(0,\infty)$
		\item [-] Range of the logarithmic function = domain of the exponential function = $(-\infty,\infty)$
	\end{enumerate}
}

\ex{5: Find the domain of each logarithmic function:}{
	\begin{enumerate}
		\item $f$($x$) = $\log_{2}(x+3) \rightarrow	x+3>0$
		\begin{itemize}
			\item $x>-3$ or $(-3,\infty)$
		\end{itemize}
		\item $g$($x$) = $\log_{b}\left(\frac{1+x}{1-x}\right) \rightarrow \frac{1+x}{1-x}>0$
		\begin{itemize}
			\item $x\neq 1,-1$.
			Now use a number line to find out where it applies.
			In this case it is $-1<x<1$ or $(-1,1)$ or ${x|x\neq 1,-1}$
		\end{itemize}
		\item $h$($x$) = $\log_{\frac{1}{2}}\abs{x} \rightarrow \abs{x}>0$
		\begin{itemize}
			\item $\textbf{Domain} = \mathbb{R}$ where $x\neq 0$, or All Real Numbers where $x\neq 0$, or ${x|x\neq 0}$
		\end{itemize}
	\end{enumerate}
}


\subsection{Natural Logarithm}
\thm{If the base of a logarithmic function is the number e, then we have the natural logarithm function. That is,}{
\begin{enumerate}
	\item y = ln x if and only if x = ey
	\item y = ln x and y = ex are inverse functions	
\end{enumerate}}
\thm{Common Logarithm Function}{If the base of a logarithmic function is the number 10, then we have the common logarithm function. If the base a of the logarithmic function is not indicated, it is understood to be 10. That is,
\begin{itemize}
	\item $y=log_{x}$ if and only if $x=10^y$
\end{itemize}}

\ex{6 \& 7: Determine the domain, range, and vertical asymptote of each logarithmic function. List any transformations.}{
	\begin{enumerate}
		\item [a.] $f(x) = \ln(x) \rightarrow g(x) = -\ln(x+2)$
		\begin{itemize}
			\item Domain: $x>-2$
			\item Range: $(-\infty,\infty)$
			\item Vertical Asymptote: $x\neq-2$
		\end{itemize} 
		\nt{The negtive applied to the natural log, seen in the equation $-\ln(x+2)$, is causing it to reflect over the x-axis.}
		\item [b.] $f(x) = \log(x) \rightarrow g(x) = 3\log(-x)-1$
	\end{enumerate}
}
\thm{Equations that contain logarithms are called logarithmic equations. Be sure to check each solution in the original equation and discard any extraneous solutions. Remember in logaM, a and M are positive and $a\neq 1$.
\begin{enumerate}
	\item Change the logarithmic equation to an exponential equation and solve for x
	\item If the exponential equation has base e, change it to the natural logarithm function
	\item If the exponential equation has base 10, change it to the common logarithm function
\end{enumerate}}

\ex{8: Solve for x}{
	\begin{enumerate}
		\item $\log_{3}(4x-7)=2$
		\begin{itemize}
			\item $3^{2}=4x-7$
			\item $9=4x-7$
			\item $x=4$
		\end{itemize}
		\item $\log_{x}(64)=2$
		\begin{itemize}
			\item $x^{2}=64$
			\item $x=\sqrt[2]{64}$
			\item $x=8$ Note: -8 does not work as a solution as base values for a logarithm must be greater than 1.
		\end{itemize}
	\end{enumerate}
}

\ex{8.5: Solve for x. Give the exact solution then use your calculator to give the approximate solution.}{
	\begin{enumerate}
		\item $e^{2x}=5$
		\begin{itemize}
			\item $\log_{e}5=2x$
			\item $\ln5=2x$
			\item $\frac{\ln5}{2} = x$
		\end{itemize}
	\end{enumerate}
}

\ex{Additional Example:}{
	\begin{enumerate}
		\item $10^{x^{2}+2x+1}=50$
		\begin{itemize}
			\item $\log(50)=x^{2}+2x+1$
                        \item $\pm\sqrt{\log(50)} = \sqrt{(x+1)}^{2}$
			\item $\pm\sqrt{\log(50)}=x+1$
			\item $x=\pm\sqrt{\log(50)}+1$
		\end{itemize}
	\end{enumerate}
}

\begin{figure}
	\centering
	\caption{$\log_{10}x$}
	\label{fig:log_10(x)}
	\includegraphics[scale=0.25]{log_(10)x}
\end{figure}

\ex{10: The concentration of alcohol in a person's blood is measurable. 
Recent medical research suggests that the risk $R$ (given as a percent) of having an accident while driving a car can be modeled by the equation $6e^{kx}$ where x is the variable concentration of alcohol n the blood and k is a constant.
\begin{enumerate}
	\item Suppose that a concentration in the blood of 0.04 results in a 10\% risk (R=10) of an accident.
	Find the constant k in the equation.
	Graph $R=6e^{kx}$ using the k value.
	\begin{itemize}
		\item do stuff so that k=20.62.
		She literally used her calc
	\end{itemize}
	\item Using the value of k, what is the risk if the concentration is 0.17?
	\begin{itemize}
		\item uhhhh she didn't do this.
	\end{itemize}
	\item Using the same value of k, what concentration of alcohol corresponds to a risk of 100\%?
	\begin{itemize}
		\item didn't do this one either.
		apparently D is the most important.
	\end{itemize}
	\item If the law asserts that anyone with a risk of having an accident of 20\% or more should not have driving privileges, at what concentration of alcohol in the blood should a driver be arrested and charged with a DUI?
	\begin{itemize}
		\item $20=6e^{12.77x}$
		\item $\frac{10}{3}=e^{12.77x}$
		\item $\ln\left(\frac{10}{3}\right)=12.77x$
		\item $x=0.94$
	\end{itemize}
\end{enumerate}}

\section{Properties of Logarithms}

\nt{Properties of logarithms:
\begin{enumerate}
	\item Identity $\rightarrow \log_{a}1=0$ or $\log_{a}a=1$
	\item Inverse $\rightarrow \log_{b}b^{x}=x$ or $b^{\log_{b}(x)}x$
	\item Product $\rightarrow \log_{a}xy=\log = \log_{a}x+\log_{a}y$
	\item Quotient $\rightarrow \log_{a}\frac{x}{y}=\log_{a}-\log_{a}y$
	\item Equality $\rightarrow \log_{b}a=\log_{b}c \Rrightarrow a=c$
	\item Change of Base Formula $\rightarrow \log_{a}b=\frac{\log_{c}b}{\log_{c}b}$
\end{enumerate}}

\ex{1: Use properties of logarithms to find the exact value of each expression. Do not use a calculator.}{
	\begin{enumerate}
		\item $\ln e^{\sqrt{2}}$
		\begin{itemize}
			\item $\log_{e}e^{\sqrt{2}}$
			\item $\sqrt{2}\times\log_{e}e$
			\item $\sqrt{2}\times\ln{e}$
		\end{itemize}
		\item $\log_{8}16-\log_{8}2$
		\begin{itemize}
			\item $\log_{8}\frac{16}{2}$
			\item $\log_{8}8$
			\item $1$
		\end{itemize}
	\end{enumerate}
}
\ex{3:Write the expression as a sum of logarithms. Express all powers as factors.}{
	\begin{enumerate}
		\item $\log_{a}\left(x\sqrt{x^{2}+1}\right)$
		\begin{itemize}
			\item $\log_{a}x + \log_{a}\sqrt{x^{2}+1}$
			\item $\log_{a}x + \log_{a}(x^{2}+1)^{\frac{1}{2}}$
			\item $\log_{a}x + \frac{1}{2}\log_{a}(x^{2}+1)$
		\end{itemize}
	\end{enumerate}
}
\ex{4: Write the expression as a difference in logarithms. Express all powrs as factors.}{
	\begin{enumerate}
		\item $\ln\left(\frac{x^{2}}{(x-1)^{3}}\right)$
		\begin{itemize}
			\item $\ln(x^{2})-\ln(x-1)^{3}$
			\item $2\ln(x)-3\ln(x-1)$
		\end{itemize}
	\end{enumerate}
}
\ex{6: Write each of the following as a single logarithm.}{
	\begin{enumerate}
		\item $\log_{a}7+4\log_{a}3$
		\begin{itemize}
			\item $\log_{a}7+\log_{a}3^{4}$
			\item $\log_{a}(7\times 3^{4})$
			\item $\log_{a}567$
		\end{itemize}
		\item $\frac{2}{3}\ln8-\ln(3^{4}-8)$
		\begin{itemize}
			\item $\ln8^{\frac{2}{3}}-\ln(3^{4}-8)$
			\item $\ln\left(\frac{8^{\frac{2}{3}}}{3^{4}-8}\right)$
			\item $\ln\left(\frac{4}{7^{3}}\right)$
		\end{itemize}
		\item $\log _a x+\log _a 9+\log _a\left(x^2+1\right)-\log _a 5$
		\begin{itemize}
			\item blah
		\end{itemize}
	\end{enumerate}
}
\ex{7: Approximate the following. Round answers to four decimal places.}{
	\begin{enumerate}
		\item $\log_{2}27$
		\begin{itemize}
			\item $\frac{\log_{10}27}{\log_{10}2}$ Note: 10 is the common base, thus it can be omitted.
			\item Another answer could be, $\frac{\ln27}{\ln2}$
		\end{itemize}
	\end{enumerate}
}
\ex{9: Use a graphing utility to graph the following}{
	\begin{enumerate}
		\item $y=\log_{2}x$
	\end{enumerate}
}
\begin{figure}
	\caption{$\log_{2}x$}
	\label{fig:log_(2)x}
	\includegraphics[scale=0.25]{log_(2)x}
\end{figure}

\section{}
\ex{1: Solve.}{
	\begin{enumerate}
		\item $2\log_{5}x=\log_{5}9$
		\begin{itemize}
			\item $x^{2}=9$
			\item Ergo, $x = \pm 3$
		\end{itemize}
	\end{enumerate}
}
\ex{2: Solve.}{
	\begin{enumerate}
                \item $\log _{4}(x+3)+\log_{4}(2-x)=1$
		\begin{itemize}
			\item $\log_{4}(x+3)(2-x) = 1$
                        \item $4^{1} = (x+3)(2-x)$
			\item $4 = x^{2}-x+6$
			\item $x^{2}+x-2$
			\item $(x+2)(x-1)$
			\item $x=-2,1$
		\end{itemize}
	\end{enumerate}
}

\section{Interest}
\dfn{Interest Formulas}{
\nt{I = amount of interest, A = final amount, P = principal, r = interest rate (as a decimal),\\ t = time (in years), n = number of times compounded per year)}
	Formulas:
\begin{enumerate}
	\item Simple Interest: \[I=Prt\] or \[A=P+I=P+Prt=P(1+rt)\]
	\item Compound Interest: \[A=P(1+\frac{r}{n})^{nt}\]
	\item Continuous Compounding: \[A=Pe^{rt}\]
\end{enumerate}
}

\ex{1: A credit union pays interest of 8\% per annum compounded quarterly on a certain savings plan. If \$100 is deposited in such a plan and the interest is left to accumulate, how much will be in the account after a year?}{
\begin{enumerate}
	\item A = P(1+$\frac{r}{n}$)$^{nt}$
	\item A = 100(1+$\frac{0.08}{4}$)$^{4*\cdot1}$
	\item A = \$108.24
\end{enumerate}
}
\ex{2 \& 3 : Determine the final amount you invest \$1000 at an annual rate of 10\% for 5 years, compounding at the amounts shown below.}{
\begin{enumerate}
	\item A = P(1+$\frac{r}{n}$)$^{nt}$
	\item P = \$1000
	\item r = 0.1
	\item t = 5
	\begin{itemize}
		\item Annually:
		\begin{itemize}
			\item \$1610.51
		\end{itemize}
		\item Semiannually
		\begin{itemize}
			\item \$1628.89
		\end{itemize}
		\item Quarterly
		\begin{itemize}
			\item \$1638.12
		\end{itemize}
		\item Monthly
		\begin{itemize}
			\item \$1645.31
		\end{itemize}
		\item Daily
		\begin{itemize}
			\item \$1648.61
		\end{itemize}
		\item Continuously
		\begin{itemize}
			\item New Formula: $Pe^{rt}$
			\item \$1648.72
		\end{itemize}
	\end{itemize}
\end{enumerate}
}

\nt{Calculate Effective rates of return:\\
The effective rate of interest is the equivalent annual simple rate of interest that would yield the same amount of compounding after 1 year.\\
To find the effective rate of interest:
\begin{enumerate}
	\item Using the appropriate formula, find the final amount $A$.
	\item Subtract the principal $P$ from the final amount $A$ to get the interest earned $I$.
	\item Using the simple interest formula, $I = Prt$, plug in values and solve for $r$.
\end{enumerate}}

\ex{4: On January 2, 2004, \$2000 is placed in an Individual Retirement Account (IRA) that will pay interest of 7\% per annum compounded continuously.}{
\begin{enumerate}
	\item [a.] What will the IRA be worth on January 1, 2024?: \\
	\[Pe=^{rt}=2000e^{0.07\cdot20}=\$8110.40\]
	\item [b.] Effective Rate of Interest?: (use the last note for guidance) \\
	\[A=Pe^{rt}=2000e^{0.07\cdot1}=2145.02\] \\
	\[2145.02-2000=145.02=I\] \\
	\[I=Prt = 145.02=2000r \rightarrow r = 0.07251 or 7.25\%\]
\end{enumerate}
}

\dfn{Determine the Present Value of a Lump Sum of Money}{
Present Value Formulas are the compound interest and continuous compounding formulas solved for the principal $P$. So,
\[P=A(1+\frac{r}{n})^{-nt}\]
\begin{center}
	or
\end{center}
\[Ae^{-rt}\]
}

\ex{5: A zero-coupon (non-interest bearing) bond can be redeemed in 10 years for \$1000. How much should you be willing to pay for it now if you want a return of:}{
\begin{enumerate}
	\item 8\% compounded monthly?
	\begin{itemize}
		\item \[P=1000(1+\frac{0.08}{12})^{-12\cdot10}\]
		\item \[\$450.52\]
	\end{itemize}
	\item 7\% compounded continuously?
	\begin{itemize}
		\item \[P=1000e^{-0.07\cdot10}\]
		\item blah, its higher.
	\end{itemize}
\end{enumerate}
}
\nt{To determine the time required to double or triple lump sums of money, if we want to double P, A will be equal to 2p. Triple is the same thing.}
\ex{6: What rate of interest compounded annually should you seek if you want to double your investment in 5 years?}{

	\[A=P(1+\frac{r}{n})^{nt}\]
	\[2p=P(1+\frac{r}{1})^{1\cdot5}\]
	Divide by P~\[ 2=(1+\frac{r}{1})^{1 \cdot 5}\]
	5th Root~\[ \sqrt[5]{2}=1+r\]
	ANS:~\[ r=\sqrt[5]{2}-1\]
	Or:~\[ r=14.9\%\]

}

\ex{7:}{
\begin{enumerate}
	\item How long will it take for an investment to double in value if it earns 5\% interest compounded continuously?
                \[A=Pe^{rt}\]
                \[2P=Pe^{0.05t}\]
                \[2=e^{0.05t}\]
                \[\ln(2)=0.05t\]
                \[t=\frac{\ln(2)}{0.05}\]
                \[t=13.86\]
	
	\item How long will it take to triple in value?
                \[3P=Pe^{0.05t}\]
                \[t=21.97\]
\end{enumerate}
}

\section{Exponential Growth and Decay}
\dfn{Exponential Growth and Decay follow one formula}{
Many natural phenomena have been found to follow the law that an amount $A$ varies with the time $t$ according to
\[A(t)=A_{0}e^{kt}\]
Where $A_{0}$ is the initial amount at $t=0$and $k$ is a constant not equal to zero.\\
If $k>0$, then $A$ increases over time (growth). \\
If $k<0$, then $A$ decreases over time (decay).
}

\ex{1: The population of a midwestern city follows the exponential law. The population decreased from 900,000 to 800,000 from 2003 to 2005.}{
\begin{enumerate}
	\item [k:]Figure out $k$ first.
	\begin{enumerate}
		\item \[t=2 \rightarrow A(t)=A_{0}e^{kt}\]
		\item \[800,000=900,000e^{2k}\]
		\item Divide by 900,000. \[\frac{8}{9}=e^{2k}\]
		\item Convert to a log. \[\ln\frac{9}{8}=2k\]
		\item Divide by two. \[\frac{\ln\frac{8}{9}}{2}=k\]
		\item \[k=\frac{\ln\frac{8}{9}}{2}\]
		\item \[k=0.05\dots\]
	\end{enumerate}
	\item [a.] What will the population be in 2007?
	\begin{enumerate}
		\item \[A(4)=900,000e^{\frac{\ln\frac{8}{9}}{2}\cdot4}\]
		\item \[711,111\]
	\end{enumerate}
	\item [b.] When will the population be half its original amount?
	\begin{enumerate}
		\item \[450,000=900,000e^{\frac{\ln\frac{8}{9}}{2}\cdot\ t}\]
		\item \[\frac{1}{2}=e^{\frac{\ln\frac{8}{9}}{2}\cdot\ t}\]
		\item \[\ln\frac{1}{2}=\frac{\ln\frac{8}{9}}{2}\cdot\ t\]
		\item \[t=\frac{\ln\frac{1}{2}}{\frac{\ln\frac{8}{9}}{2}}\]
	\end{enumerate}
	\item [c.] When will the population reach 300,000?
	\begin{enumerate}
		\item \[\frac{1}{3}=e^{\frac{\ln\frac{8}{9}}{2}\cdot\ t}\]
		\item \[t=18.65 or 19 years\]
	\end{enumerate}
\end{enumerate}
}
\ex{2: A colony of bacteria grows according to the law of unhibited growth according to the function $N(t)=100e^{0.045t}$, where $N$ is measured in grams and $t$ is in days.}{
\begin{enumerate}
	\item Determine the initial amount of bacteria.
	\begin{enumerate}
		\item \[N(0)=100e^{0.045(0)}\]
	\end{enumerate}
	\item What is the growth rate of the bacteria?
	\begin{enumerate}
		\item \[4.5\%\]
	\end{enumerate}
	\item Graph the function using your calculator
	\begin{enumerate}
		\item Insert Image here later
	\end{enumerate}
	\item What is the population after 5 days?
	\begin{enumerate}
		\item \[N(5)=100e^{0.045(5)}\]
		\item \[N(5)=125.2\]
	\end{enumerate}
	\item How long will it take for the population to reach 140 grams?
	\begin{enumerate}
		\item \[N(t)=140 \rightarrow 140=100e^{0.045(t)}\]
		\item \[t=\frac{\ln\frac{7}{5}}{0.045}=7.5\]
	\end{enumerate}
	\item What is the doubling time for the population?
	\begin{enumerate}
		\item \[N(t)=200 \rightarrow 200=100e^{0.045(t)}\]
		\item \[t=\frac{\ln 2}{0.045}=15.4\]
	\end{enumerate}
\end{enumerate}
}

\subsection{Half-Life}\label{subsec:half-life}

\dfn{Half-Life}{
    The amount of time required for any specific property to decrease by half.
    Steps:
\begin{enumerate}
    \item Find $k$ using \[A(t)=\frac{1}{2}A_{0}\]
    \item using the percent of the original amount left, replace $A(t)$ with the percent as a decimal times $A_{0}$ meaning $A(t)=\% A_{0}$
\end{enumerate}
}

\ex{3: Traces of burning wood along with ancient stone tools in an archaeological dig in Chile were found to contain approximately 1.67\% of the original amount of Carbon-14. If the half-life of Carbon-14 is 5,600 years, approximately when was the tree cut and burned?}{
    Find $k$:
    \begin{enumerate}
    \item \[\frac{1}{2}A_{0}=A_{0}e^{k\cdot5600}\]
    \item \[\frac{\ln\frac{1}{2}}{5600}=k\]
    \end{enumerate}
    Find $t$:
\begin{enumerate}
    \item \[0.167\cdot\ A_{0}=A_{0}e^{\frac{\ln\frac{1}{2}}{5600}\cdot\ t}\]
    \item \[\frac{\ln 0.167}{\frac{\ln\frac{1}{2}}{5600}}=t\]
    \item \[t=33,062\]
\end{enumerate}
}

\dfn{Newton's Law of Cooling}{
The temperature $u$ of a heated object at a given time $t$ can be modeled by the following function:
\[u(t)=T+(u_{0}-T)e^{kt}\]
Where $T$ is the constant temperature of the surrounding medium, $u_{0}$ is the initial temperature of the heated object, and $k$ is a negative constant.
}

\ex{4:}{
	An object is heated to 100C and is then allowed to cool in a room whose air temperature is 30C?
	\[u(t)=T+(u_{0}-T)e^{kt}\]
	\begin{enumerate}
		\item Find $k$:
		\[80=30+(100-30)e^{k\cdot\ 5}\]
		\[80=30+70e^{5k}\]
		\[\frac{50}{70}=e^{5k}\]
		\[\frac{5}{7}=e^{5k}\]
		\[k=\frac{\ln\frac{5}{7}}{5}\]
		\item If the temperature of the object is 80C after 5 minutes, when will the temperature be 50C?
		\[50=30+(100-30)e^{\frac{\ln\frac{5}{7}}{5}\cdot\ t}\]
		\[t=18.616\]
		\item What is the temperature after 30 minutes?
		\[u(30)=30+(100-30)e^{\frac{\ln\frac{5}{7}}{5}\cdot\ 30}\]
		\[u=39.3\]
	\end{enumerate}
}

\dfn{Logistic Model}{
In a logistic growth model, the population $P$ after time $t$ obeys the equation:
\[P(t)=\frac{c}{1+ae^{-bt}}\]
Where $a$,$b$, and $c$ are constants with $c>0$
The model is a growth model if $b>0$
The model is a decay model if $b<0$
}

\ex{5:}{
    Fruit flies are placed in a half-pint milk bottle with a banana (food) and yeast plants (for food and to provide a stimulus to lay eggs).
    Suppose the fruit fly population after $t$ days is given by:
\[P(t)=\frac{230}{1+56.5e^{-0.37\cdot\ t}}\]
\begin{enumerate}
    \item State the carrying capacity and growth rate.
    \begin{enumerate}
        \item c=230, b=0.37 or 37\%
    \end{enumerate}
    \item Determine the initial population.
    \begin{enumerate}
        \item t=o
        \item \[\frac{230}{1+5.65}\]
    \end{enumerate}
    \item What is the population after 5 days?
    \begin{enumerate}
        \item $t=$
        \item \[\frac{230}{1+5.65e^{-0.37\cdot\ 5}}\]
        \item \[P(t)=23\]
    \end{enumerate}
    \item How long does it take for the population to reach 180?
    \begin{enumerate}
        \item $P(t)=180$
        \item \[180=\frac{230}{1+5.65e^{-0.37\cdot\ t}}\]
        \item \[180(1+5.65e^{-0.37\cdot\ t})= 230\]
        \item \[t=14.36\]
    \end{enumerate}
    \item What do you notice about the population as time passes?
    \begin{enumerate}
        \item It will approach 230.
    \end{enumerate}
\end{enumerate}
}




$\lim_{x \to 1}$



















\newpage
\begin{figure}
	\centering
	\Huge{Thanks for reading}
	\includegraphics[scale=0.2]{dabloon}
	\label{fig:dabloonia}
\end{figure}




\end{document}
